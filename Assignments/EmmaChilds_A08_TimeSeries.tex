% Options for packages loaded elsewhere
\PassOptionsToPackage{unicode}{hyperref}
\PassOptionsToPackage{hyphens}{url}
%
\documentclass[
]{article}
\usepackage{amsmath,amssymb}
\usepackage{lmodern}
\usepackage{iftex}
\ifPDFTeX
  \usepackage[T1]{fontenc}
  \usepackage[utf8]{inputenc}
  \usepackage{textcomp} % provide euro and other symbols
\else % if luatex or xetex
  \usepackage{unicode-math}
  \defaultfontfeatures{Scale=MatchLowercase}
  \defaultfontfeatures[\rmfamily]{Ligatures=TeX,Scale=1}
\fi
% Use upquote if available, for straight quotes in verbatim environments
\IfFileExists{upquote.sty}{\usepackage{upquote}}{}
\IfFileExists{microtype.sty}{% use microtype if available
  \usepackage[]{microtype}
  \UseMicrotypeSet[protrusion]{basicmath} % disable protrusion for tt fonts
}{}
\makeatletter
\@ifundefined{KOMAClassName}{% if non-KOMA class
  \IfFileExists{parskip.sty}{%
    \usepackage{parskip}
  }{% else
    \setlength{\parindent}{0pt}
    \setlength{\parskip}{6pt plus 2pt minus 1pt}}
}{% if KOMA class
  \KOMAoptions{parskip=half}}
\makeatother
\usepackage{xcolor}
\usepackage[margin=2.54cm]{geometry}
\usepackage{color}
\usepackage{fancyvrb}
\newcommand{\VerbBar}{|}
\newcommand{\VERB}{\Verb[commandchars=\\\{\}]}
\DefineVerbatimEnvironment{Highlighting}{Verbatim}{commandchars=\\\{\}}
% Add ',fontsize=\small' for more characters per line
\usepackage{framed}
\definecolor{shadecolor}{RGB}{248,248,248}
\newenvironment{Shaded}{\begin{snugshade}}{\end{snugshade}}
\newcommand{\AlertTok}[1]{\textcolor[rgb]{0.94,0.16,0.16}{#1}}
\newcommand{\AnnotationTok}[1]{\textcolor[rgb]{0.56,0.35,0.01}{\textbf{\textit{#1}}}}
\newcommand{\AttributeTok}[1]{\textcolor[rgb]{0.77,0.63,0.00}{#1}}
\newcommand{\BaseNTok}[1]{\textcolor[rgb]{0.00,0.00,0.81}{#1}}
\newcommand{\BuiltInTok}[1]{#1}
\newcommand{\CharTok}[1]{\textcolor[rgb]{0.31,0.60,0.02}{#1}}
\newcommand{\CommentTok}[1]{\textcolor[rgb]{0.56,0.35,0.01}{\textit{#1}}}
\newcommand{\CommentVarTok}[1]{\textcolor[rgb]{0.56,0.35,0.01}{\textbf{\textit{#1}}}}
\newcommand{\ConstantTok}[1]{\textcolor[rgb]{0.00,0.00,0.00}{#1}}
\newcommand{\ControlFlowTok}[1]{\textcolor[rgb]{0.13,0.29,0.53}{\textbf{#1}}}
\newcommand{\DataTypeTok}[1]{\textcolor[rgb]{0.13,0.29,0.53}{#1}}
\newcommand{\DecValTok}[1]{\textcolor[rgb]{0.00,0.00,0.81}{#1}}
\newcommand{\DocumentationTok}[1]{\textcolor[rgb]{0.56,0.35,0.01}{\textbf{\textit{#1}}}}
\newcommand{\ErrorTok}[1]{\textcolor[rgb]{0.64,0.00,0.00}{\textbf{#1}}}
\newcommand{\ExtensionTok}[1]{#1}
\newcommand{\FloatTok}[1]{\textcolor[rgb]{0.00,0.00,0.81}{#1}}
\newcommand{\FunctionTok}[1]{\textcolor[rgb]{0.00,0.00,0.00}{#1}}
\newcommand{\ImportTok}[1]{#1}
\newcommand{\InformationTok}[1]{\textcolor[rgb]{0.56,0.35,0.01}{\textbf{\textit{#1}}}}
\newcommand{\KeywordTok}[1]{\textcolor[rgb]{0.13,0.29,0.53}{\textbf{#1}}}
\newcommand{\NormalTok}[1]{#1}
\newcommand{\OperatorTok}[1]{\textcolor[rgb]{0.81,0.36,0.00}{\textbf{#1}}}
\newcommand{\OtherTok}[1]{\textcolor[rgb]{0.56,0.35,0.01}{#1}}
\newcommand{\PreprocessorTok}[1]{\textcolor[rgb]{0.56,0.35,0.01}{\textit{#1}}}
\newcommand{\RegionMarkerTok}[1]{#1}
\newcommand{\SpecialCharTok}[1]{\textcolor[rgb]{0.00,0.00,0.00}{#1}}
\newcommand{\SpecialStringTok}[1]{\textcolor[rgb]{0.31,0.60,0.02}{#1}}
\newcommand{\StringTok}[1]{\textcolor[rgb]{0.31,0.60,0.02}{#1}}
\newcommand{\VariableTok}[1]{\textcolor[rgb]{0.00,0.00,0.00}{#1}}
\newcommand{\VerbatimStringTok}[1]{\textcolor[rgb]{0.31,0.60,0.02}{#1}}
\newcommand{\WarningTok}[1]{\textcolor[rgb]{0.56,0.35,0.01}{\textbf{\textit{#1}}}}
\usepackage{graphicx}
\makeatletter
\def\maxwidth{\ifdim\Gin@nat@width>\linewidth\linewidth\else\Gin@nat@width\fi}
\def\maxheight{\ifdim\Gin@nat@height>\textheight\textheight\else\Gin@nat@height\fi}
\makeatother
% Scale images if necessary, so that they will not overflow the page
% margins by default, and it is still possible to overwrite the defaults
% using explicit options in \includegraphics[width, height, ...]{}
\setkeys{Gin}{width=\maxwidth,height=\maxheight,keepaspectratio}
% Set default figure placement to htbp
\makeatletter
\def\fps@figure{htbp}
\makeatother
\setlength{\emergencystretch}{3em} % prevent overfull lines
\providecommand{\tightlist}{%
  \setlength{\itemsep}{0pt}\setlength{\parskip}{0pt}}
\setcounter{secnumdepth}{-\maxdimen} % remove section numbering
\ifLuaTeX
  \usepackage{selnolig}  % disable illegal ligatures
\fi
\IfFileExists{bookmark.sty}{\usepackage{bookmark}}{\usepackage{hyperref}}
\IfFileExists{xurl.sty}{\usepackage{xurl}}{} % add URL line breaks if available
\urlstyle{same} % disable monospaced font for URLs
\hypersetup{
  pdftitle={Assignment 8: Time Series Analysis},
  pdfauthor={Emma Childs},
  hidelinks,
  pdfcreator={LaTeX via pandoc}}

\title{Assignment 8: Time Series Analysis}
\author{Emma Childs}
\date{Spring 2023}

\begin{document}
\maketitle

\hypertarget{overview}{%
\subsection{OVERVIEW}\label{overview}}

This exercise accompanies the lessons in Environmental Data Analytics on
generalized linear models.

\hypertarget{directions}{%
\subsection{Directions}\label{directions}}

\begin{enumerate}
\def\labelenumi{\arabic{enumi}.}
\tightlist
\item
  Rename this file
  \texttt{\textless{}FirstLast\textgreater{}\_A08\_TimeSeries.Rmd}
  (replacing \texttt{\textless{}FirstLast\textgreater{}} with your first
  and last name).
\item
  Change ``Student Name'' on line 3 (above) with your name.
\item
  Work through the steps, \textbf{creating code and output} that fulfill
  each instruction.
\item
  Be sure to \textbf{answer the questions} in this assignment document.
\item
  When you have completed the assignment, \textbf{Knit} the text and
  code into a single PDF file.
\end{enumerate}

\hypertarget{set-up}{%
\subsection{Set up}\label{set-up}}

\begin{enumerate}
\def\labelenumi{\arabic{enumi}.}
\tightlist
\item
  Set up your session:
\end{enumerate}

\begin{itemize}
\tightlist
\item
  Check your working directory
\item
  Load the tidyverse, lubridate, zoo, and trend packages
\item
  Set your ggplot theme
\end{itemize}

\begin{Shaded}
\begin{Highlighting}[]
\CommentTok{\#1}
\FunctionTok{getwd}\NormalTok{()}
\end{Highlighting}
\end{Shaded}

\begin{verbatim}
## [1] "/home/guest/EDA-Spring2023"
\end{verbatim}

\begin{Shaded}
\begin{Highlighting}[]
\FunctionTok{library}\NormalTok{(tidyverse)}
\end{Highlighting}
\end{Shaded}

\begin{verbatim}
## -- Attaching core tidyverse packages ------------------------ tidyverse 2.0.0 --
## v dplyr     1.1.0     v readr     2.1.4
## v forcats   1.0.0     v stringr   1.5.0
## v ggplot2   3.4.1     v tibble    3.1.8
## v lubridate 1.9.2     v tidyr     1.3.0
## v purrr     1.0.1     
## -- Conflicts ------------------------------------------ tidyverse_conflicts() --
## x dplyr::filter() masks stats::filter()
## x dplyr::lag()    masks stats::lag()
## i Use the ]8;;http://conflicted.r-lib.org/conflicted package]8;; to force all conflicts to become errors
\end{verbatim}

\begin{Shaded}
\begin{Highlighting}[]
\FunctionTok{library}\NormalTok{(lubridate)}
\FunctionTok{install.packages}\NormalTok{(}\StringTok{"trend"}\NormalTok{)}
\end{Highlighting}
\end{Shaded}

\begin{verbatim}
## Installing package into '/home/guest/R/x86_64-pc-linux-gnu-library/4.2'
## (as 'lib' is unspecified)
\end{verbatim}

\begin{Shaded}
\begin{Highlighting}[]
\FunctionTok{library}\NormalTok{(trend)}
\FunctionTok{install.packages}\NormalTok{(}\StringTok{"zoo"}\NormalTok{)}
\end{Highlighting}
\end{Shaded}

\begin{verbatim}
## Installing package into '/home/guest/R/x86_64-pc-linux-gnu-library/4.2'
## (as 'lib' is unspecified)
\end{verbatim}

\begin{Shaded}
\begin{Highlighting}[]
\FunctionTok{library}\NormalTok{(zoo)}
\end{Highlighting}
\end{Shaded}

\begin{verbatim}
## 
## Attaching package: 'zoo'
## 
## The following objects are masked from 'package:base':
## 
##     as.Date, as.Date.numeric
\end{verbatim}

\begin{Shaded}
\begin{Highlighting}[]
\FunctionTok{library}\NormalTok{(dplyr)}
\FunctionTok{library}\NormalTok{(scales)}
\end{Highlighting}
\end{Shaded}

\begin{verbatim}
## 
## Attaching package: 'scales'
## 
## The following object is masked from 'package:purrr':
## 
##     discard
## 
## The following object is masked from 'package:readr':
## 
##     col_factor
\end{verbatim}

\begin{Shaded}
\begin{Highlighting}[]
\FunctionTok{library}\NormalTok{(here)}
\end{Highlighting}
\end{Shaded}

\begin{verbatim}
## here() starts at /home/guest/EDA-Spring2023
\end{verbatim}

\begin{Shaded}
\begin{Highlighting}[]
\FunctionTok{library}\NormalTok{(Kendall)}

\NormalTok{mytheme }\OtherTok{\textless{}{-}} \FunctionTok{theme\_classic}\NormalTok{(}\AttributeTok{base\_size =} \DecValTok{12}\NormalTok{) }\SpecialCharTok{+}
  \FunctionTok{theme}\NormalTok{(}\AttributeTok{axis.text =} \FunctionTok{element\_text}\NormalTok{(}\AttributeTok{color =} \StringTok{"black"}\NormalTok{), }
        \AttributeTok{legend.position =} \StringTok{"top"}\NormalTok{)}
\FunctionTok{theme\_set}\NormalTok{(mytheme)}
\end{Highlighting}
\end{Shaded}

\begin{enumerate}
\def\labelenumi{\arabic{enumi}.}
\setcounter{enumi}{1}
\tightlist
\item
  Import the ten datasets from the Ozone\_TimeSeries folder in the Raw
  data folder. These contain ozone concentrations at Garinger High
  School in North Carolina from 2010-2019 (the EPA air database only
  allows downloads for one year at a time). Import these either
  individually or in bulk and then combine them into a single dataframe
  named \texttt{GaringerOzone} of 3589 observation and 20 variables.
\end{enumerate}

\begin{Shaded}
\begin{Highlighting}[]
\CommentTok{\#2 Importing Data}
\NormalTok{Garinger2010 }\OtherTok{\textless{}{-}} \FunctionTok{read.csv}\NormalTok{(}\FunctionTok{here}\NormalTok{(}\StringTok{"Data/Raw/Ozone\_TimeSeries/EPAair\_O3\_GaringerNC2010\_raw.csv"}\NormalTok{), }\AttributeTok{stringsAsFactors =} \ConstantTok{TRUE}\NormalTok{)}
\NormalTok{Garinger2011 }\OtherTok{\textless{}{-}} \FunctionTok{read.csv}\NormalTok{(}\FunctionTok{here}\NormalTok{(}\StringTok{"Data/Raw/Ozone\_TimeSeries/EPAair\_O3\_GaringerNC2011\_raw.csv"}\NormalTok{), }\AttributeTok{stringsAsFactors =} \ConstantTok{TRUE}\NormalTok{)}
\NormalTok{Garinger2012 }\OtherTok{\textless{}{-}} \FunctionTok{read.csv}\NormalTok{(}\FunctionTok{here}\NormalTok{(}\StringTok{"Data/Raw/Ozone\_TimeSeries/EPAair\_O3\_GaringerNC2012\_raw.csv"}\NormalTok{), }\AttributeTok{stringsAsFactors =} \ConstantTok{TRUE}\NormalTok{)}
\NormalTok{Garinger2013 }\OtherTok{\textless{}{-}} \FunctionTok{read.csv}\NormalTok{(}\FunctionTok{here}\NormalTok{(}\StringTok{"Data/Raw/Ozone\_TimeSeries/EPAair\_O3\_GaringerNC2013\_raw.csv"}\NormalTok{), }\AttributeTok{stringsAsFactors =} \ConstantTok{TRUE}\NormalTok{)}
\NormalTok{Garinger2014 }\OtherTok{\textless{}{-}} \FunctionTok{read.csv}\NormalTok{(}\FunctionTok{here}\NormalTok{(}\StringTok{"Data/Raw/Ozone\_TimeSeries/EPAair\_O3\_GaringerNC2014\_raw.csv"}\NormalTok{), }\AttributeTok{stringsAsFactors =} \ConstantTok{TRUE}\NormalTok{)}
\NormalTok{Garinger2015 }\OtherTok{\textless{}{-}} \FunctionTok{read.csv}\NormalTok{(}\FunctionTok{here}\NormalTok{(}\StringTok{"Data/Raw/Ozone\_TimeSeries/EPAair\_O3\_GaringerNC2015\_raw.csv"}\NormalTok{), }\AttributeTok{stringsAsFactors =} \ConstantTok{TRUE}\NormalTok{)}
\NormalTok{Garinger2016 }\OtherTok{\textless{}{-}} \FunctionTok{read.csv}\NormalTok{(}\FunctionTok{here}\NormalTok{(}\StringTok{"Data/Raw/Ozone\_TimeSeries/EPAair\_O3\_GaringerNC2016\_raw.csv"}\NormalTok{), }\AttributeTok{stringsAsFactors =} \ConstantTok{TRUE}\NormalTok{)}
\NormalTok{Garinger2017 }\OtherTok{\textless{}{-}} \FunctionTok{read.csv}\NormalTok{(}\FunctionTok{here}\NormalTok{(}\StringTok{"Data/Raw/Ozone\_TimeSeries/EPAair\_O3\_GaringerNC2017\_raw.csv"}\NormalTok{), }\AttributeTok{stringsAsFactors =} \ConstantTok{TRUE}\NormalTok{)}
\NormalTok{Garinger2018 }\OtherTok{\textless{}{-}} \FunctionTok{read.csv}\NormalTok{(}\FunctionTok{here}\NormalTok{(}\StringTok{"Data/Raw/Ozone\_TimeSeries/EPAair\_O3\_GaringerNC2018\_raw.csv"}\NormalTok{), }\AttributeTok{stringsAsFactors =} \ConstantTok{TRUE}\NormalTok{)}
\NormalTok{Garinger2019 }\OtherTok{\textless{}{-}} \FunctionTok{read.csv}\NormalTok{(}\FunctionTok{here}\NormalTok{(}\StringTok{"Data/Raw/Ozone\_TimeSeries/EPAair\_O3\_GaringerNC2019\_raw.csv"}\NormalTok{), }\AttributeTok{stringsAsFactors =} \ConstantTok{TRUE}\NormalTok{)}

\NormalTok{GaringerOzone }\OtherTok{\textless{}{-}} \FunctionTok{rbind}\NormalTok{(Garinger2010, Garinger2011, Garinger2012, Garinger2013, Garinger2014, Garinger2015, Garinger2016, Garinger2017, Garinger2018, Garinger2019)}
\end{Highlighting}
\end{Shaded}

\hypertarget{wrangle}{%
\subsection{Wrangle}\label{wrangle}}

\begin{enumerate}
\def\labelenumi{\arabic{enumi}.}
\setcounter{enumi}{2}
\item
  Set your date column as a date class.
\item
  Wrangle your dataset so that it only contains the columns Date,
  Daily.Max.8.hour.Ozone.Concentration, and DAILY\_AQI\_VALUE.
\item
  Notice there are a few days in each year that are missing ozone
  concentrations. We want to generate a daily dataset, so we will need
  to fill in any missing days with NA. Create a new data frame that
  contains a sequence of dates from 2010-01-01 to 2019-12-31 (hint:
  \texttt{as.data.frame(seq())}). Call this new data frame Days. Rename
  the column name in Days to ``Date''.
\item
  Use a \texttt{left\_join} to combine the data frames. Specify the
  correct order of data frames within this function so that the final
  dimensions are 3652 rows and 3 columns. Call your combined data frame
  GaringerOzone.
\end{enumerate}

\begin{Shaded}
\begin{Highlighting}[]
\CommentTok{\#3 }
\NormalTok{GaringerOzone}\SpecialCharTok{$}\NormalTok{Date }\OtherTok{\textless{}{-}} \FunctionTok{mdy}\NormalTok{(GaringerOzone}\SpecialCharTok{$}\NormalTok{Date)}
\CommentTok{\#turning date column to date class}

\CommentTok{\#4}
\NormalTok{GaringerOzone\_subset }\OtherTok{\textless{}{-}}
\NormalTok{  GaringerOzone }\SpecialCharTok{\%\textgreater{}\%}
  \FunctionTok{select}\NormalTok{(Date, Daily.Max.}\FloatTok{8.}\NormalTok{hour.Ozone.Concentration, DAILY\_AQI\_VALUE)}
\CommentTok{\#subsetting data to just include these 4 columns}

\FunctionTok{summary}\NormalTok{(GaringerOzone\_subset}\SpecialCharTok{$}\NormalTok{Daily.Max.}\FloatTok{8.}\NormalTok{hour.Ozone.Concentration)}
\end{Highlighting}
\end{Shaded}

\begin{verbatim}
##    Min. 1st Qu.  Median    Mean 3rd Qu.    Max. 
## 0.00200 0.03200 0.04100 0.04163 0.05100 0.09300
\end{verbatim}

\begin{Shaded}
\begin{Highlighting}[]
\CommentTok{\#trying to find missing values {-} not sure if it was successful}

\CommentTok{\#5}
\NormalTok{Days }\OtherTok{\textless{}{-}} \FunctionTok{as.data.frame}\NormalTok{(}\FunctionTok{seq}\NormalTok{(}\FunctionTok{as.Date}\NormalTok{(}\StringTok{"2010{-}01{-}01"}\NormalTok{), }\FunctionTok{as.Date}\NormalTok{(}\StringTok{"2019{-}12{-}31"}\NormalTok{), }\AttributeTok{by =} \StringTok{"1 day"}\NormalTok{))}
\FunctionTok{colnames}\NormalTok{(Days) }\OtherTok{\textless{}{-}} \StringTok{"Date"}
\CommentTok{\#creating daily dataset, filling in missing values with NA}

\CommentTok{\#6}
\NormalTok{GaringerOzone\_new }\OtherTok{\textless{}{-}} \FunctionTok{left\_join}\NormalTok{(Days, GaringerOzone\_subset)}
\end{Highlighting}
\end{Shaded}

\begin{verbatim}
## Joining with `by = join_by(Date)`
\end{verbatim}

\hypertarget{visualize}{%
\subsection{Visualize}\label{visualize}}

\begin{enumerate}
\def\labelenumi{\arabic{enumi}.}
\setcounter{enumi}{6}
\tightlist
\item
  Create a line plot depicting ozone concentrations over time. In this
  case, we will plot actual concentrations in ppm, not AQI values.
  Format your axes accordingly. Add a smoothed line showing any linear
  trend of your data. Does your plot suggest a trend in ozone
  concentration over time?
\end{enumerate}

\begin{Shaded}
\begin{Highlighting}[]
\CommentTok{\#7}
\NormalTok{OzoneOverTime\_Line }\OtherTok{\textless{}{-}} \FunctionTok{ggplot}\NormalTok{(GaringerOzone\_new, }
    \FunctionTok{aes}\NormalTok{(}\AttributeTok{x =}\NormalTok{ Date, }\AttributeTok{y =}\NormalTok{ Daily.Max.}\FloatTok{8.}\NormalTok{hour.Ozone.Concentration)) }\SpecialCharTok{+}
  \FunctionTok{scale\_x\_date}\NormalTok{(}\AttributeTok{labels =} \FunctionTok{date\_format}\NormalTok{(}\StringTok{"\%Y"}\NormalTok{), }\AttributeTok{date\_breaks =} \StringTok{"1 year"}\NormalTok{) }\SpecialCharTok{+} 
      \FunctionTok{geom\_line}\NormalTok{() }\SpecialCharTok{+}
    \FunctionTok{geom\_smooth}\NormalTok{(}\AttributeTok{method =}\NormalTok{ lm, }\AttributeTok{color =} \StringTok{"purple"}\NormalTok{, }\AttributeTok{se =}\NormalTok{ F) }\SpecialCharTok{+}
    \FunctionTok{labs}\NormalTok{(}\AttributeTok{y =} \StringTok{"Ozone Concentration (ppm)"}\NormalTok{)}
\FunctionTok{print}\NormalTok{(OzoneOverTime\_Line)}
\end{Highlighting}
\end{Shaded}

\begin{verbatim}
## `geom_smooth()` using formula = 'y ~ x'
\end{verbatim}

\begin{verbatim}
## Warning: Removed 63 rows containing non-finite values (`stat_smooth()`).
\end{verbatim}

\includegraphics{EmmaChilds_A08_TimeSeries_files/figure-latex/unnamed-chunk-4-1.pdf}

\begin{quote}
Answer: The data looks like there might be slight linear decrease in
ozone concentration over time, but it seems very slight, and without
statistical analysis, I don't think we can verify if it is significant.
But using the eyeball test of sorts, it looks like it is decreasing
slightly.
\end{quote}

\hypertarget{time-series-analysis}{%
\subsection{Time Series Analysis}\label{time-series-analysis}}

Study question: Have ozone concentrations changed over the 2010s at this
station?

\begin{enumerate}
\def\labelenumi{\arabic{enumi}.}
\setcounter{enumi}{7}
\tightlist
\item
  Use a linear interpolation to fill in missing daily data for ozone
  concentration. Why didn't we use a piecewise constant or spline
  interpolation?
\end{enumerate}

\begin{Shaded}
\begin{Highlighting}[]
\CommentTok{\#8}
\FunctionTok{summary}\NormalTok{(GaringerOzone\_new}\SpecialCharTok{$}\NormalTok{Daily.Max.}\FloatTok{8.}\NormalTok{hour.Ozone.Concentration)}
\end{Highlighting}
\end{Shaded}

\begin{verbatim}
##    Min. 1st Qu.  Median    Mean 3rd Qu.    Max.    NA's 
## 0.00200 0.03200 0.04100 0.04163 0.05100 0.09300      63
\end{verbatim}

\begin{Shaded}
\begin{Highlighting}[]
\NormalTok{GaringerOzone\_new}\SpecialCharTok{$}\NormalTok{Daily.Max.}\FloatTok{8.}\NormalTok{hour.Ozone.Concentration }\OtherTok{\textless{}{-}}
\FunctionTok{na.approx}\NormalTok{(GaringerOzone\_new}\SpecialCharTok{$}\NormalTok{Daily.Max.}\FloatTok{8.}\NormalTok{hour.Ozone.Concentration)}
\end{Highlighting}
\end{Shaded}

\begin{quote}
Answer: We use a linear interpolation to fill in this missing data for
ozone concentration because the graphed data depict a linear trend. This
helps account for the na's that we estimated/approximated.
\end{quote}

\begin{enumerate}
\def\labelenumi{\arabic{enumi}.}
\setcounter{enumi}{8}
\tightlist
\item
  Create a new data frame called \texttt{GaringerOzone.monthly} that
  contains aggregated data: mean ozone concentrations for each month. In
  your pipe, you will need to first add columns for year and month to
  form the groupings. In a separate line of code, create a new Date
  column with each month-year combination being set as the first day of
  the month (this is for graphing purposes only)
\end{enumerate}

\begin{Shaded}
\begin{Highlighting}[]
\CommentTok{\#9}
\NormalTok{GaringerOzone.monthly }\OtherTok{\textless{}{-}}
\NormalTok{GaringerOzone\_new }\SpecialCharTok{\%\textgreater{}\%}
\FunctionTok{mutate}\NormalTok{(}\AttributeTok{month =} \FunctionTok{month}\NormalTok{(Date)) }\SpecialCharTok{\%\textgreater{}\%} 
\FunctionTok{mutate}\NormalTok{(}\AttributeTok{year =} \FunctionTok{year}\NormalTok{(Date)) }\SpecialCharTok{\%\textgreater{}\%}
\FunctionTok{group\_by}\NormalTok{(year,month) }\SpecialCharTok{\%\textgreater{}\%}  
  \FunctionTok{summarise}\NormalTok{(}\AttributeTok{meanOzoneConcentration =} \FunctionTok{mean}\NormalTok{(Daily.Max.}\FloatTok{8.}\NormalTok{hour.Ozone.Concentration))}
\end{Highlighting}
\end{Shaded}

\begin{verbatim}
## `summarise()` has grouped output by 'year'. You can override using the
## `.groups` argument.
\end{verbatim}

\begin{Shaded}
\begin{Highlighting}[]
\NormalTok{GaringerOzone.monthly }\OtherTok{\textless{}{-}}
\NormalTok{GaringerOzone.monthly }\SpecialCharTok{\%\textgreater{}\%} \FunctionTok{mutate}\NormalTok{(}\StringTok{"firstmonth"}\OtherTok{=}\FunctionTok{my}\NormalTok{(}\FunctionTok{paste0}\NormalTok{(month,}\StringTok{"{-}"}\NormalTok{,year)))}
\end{Highlighting}
\end{Shaded}

\begin{enumerate}
\def\labelenumi{\arabic{enumi}.}
\setcounter{enumi}{9}
\tightlist
\item
  Generate two time series objects. Name the first
  \texttt{GaringerOzone.daily.ts} and base it on the dataframe of daily
  observations. Name the second \texttt{GaringerOzone.monthly.ts} and
  base it on the monthly average ozone values. Be sure that each
  specifies the correct start and end dates and the frequency of the
  time series.
\end{enumerate}

\begin{Shaded}
\begin{Highlighting}[]
\CommentTok{\#10}
\NormalTok{GaringerOzone.daily.ts }\OtherTok{\textless{}{-}} \FunctionTok{ts}\NormalTok{(GaringerOzone\_new}\SpecialCharTok{$}\NormalTok{Daily.Max.}\FloatTok{8.}\NormalTok{hour.Ozone.Concentration, }\AttributeTok{start =} \FunctionTok{c}\NormalTok{(}\DecValTok{2010}\NormalTok{,}\DecValTok{1}\NormalTok{), }\AttributeTok{frequency =} \DecValTok{365}\NormalTok{)}
\NormalTok{GaringerOzone.monthly.ts }\OtherTok{\textless{}{-}} \FunctionTok{ts}\NormalTok{(GaringerOzone.monthly}\SpecialCharTok{$}\NormalTok{meanOzoneConcentration, }\AttributeTok{start =} \FunctionTok{c}\NormalTok{(}\DecValTok{2010}\NormalTok{,}\DecValTok{1}\NormalTok{), }\AttributeTok{frequency =} \DecValTok{12}\NormalTok{)}
\end{Highlighting}
\end{Shaded}

\begin{enumerate}
\def\labelenumi{\arabic{enumi}.}
\setcounter{enumi}{10}
\tightlist
\item
  Decompose the daily and the monthly time series objects and plot the
  components using the \texttt{plot()} function.
\end{enumerate}

\begin{Shaded}
\begin{Highlighting}[]
\CommentTok{\#11}
\NormalTok{GOzone\_daily\_decomposed }\OtherTok{\textless{}{-}} \FunctionTok{stl}\NormalTok{(GaringerOzone.daily.ts, }\AttributeTok{s.window =} \StringTok{"periodic"}\NormalTok{)}
\FunctionTok{plot}\NormalTok{(GOzone\_daily\_decomposed)}
\end{Highlighting}
\end{Shaded}

\includegraphics{EmmaChilds_A08_TimeSeries_files/figure-latex/unnamed-chunk-8-1.pdf}

\begin{Shaded}
\begin{Highlighting}[]
\NormalTok{GOzone\_monthly\_decomposed }\OtherTok{\textless{}{-}} \FunctionTok{stl}\NormalTok{(GaringerOzone.monthly.ts, }\AttributeTok{s.window =} \StringTok{"periodic"}\NormalTok{)}
\FunctionTok{plot}\NormalTok{(GOzone\_monthly\_decomposed)}
\end{Highlighting}
\end{Shaded}

\includegraphics{EmmaChilds_A08_TimeSeries_files/figure-latex/unnamed-chunk-8-2.pdf}

\begin{enumerate}
\def\labelenumi{\arabic{enumi}.}
\setcounter{enumi}{11}
\tightlist
\item
  Run a monotonic trend analysis for the monthly Ozone series. In this
  case the seasonal Mann-Kendall is most appropriate; why is this?
\end{enumerate}

\begin{Shaded}
\begin{Highlighting}[]
\CommentTok{\#12}
\NormalTok{Garinger\_monthly\_seasonal }\OtherTok{\textless{}{-}}\NormalTok{ trend}\SpecialCharTok{::}\FunctionTok{smk.test}\NormalTok{(GaringerOzone.monthly.ts)}
\end{Highlighting}
\end{Shaded}

\begin{quote}
Answer: The seasonal Mann-Kendall is FINISH EMMA
\end{quote}

\begin{enumerate}
\def\labelenumi{\arabic{enumi}.}
\setcounter{enumi}{12}
\tightlist
\item
  Create a plot depicting mean monthly ozone concentrations over time,
  with both a geom\_point and a geom\_line layer. Edit your axis labels
  accordingly.
\end{enumerate}

\begin{Shaded}
\begin{Highlighting}[]
\CommentTok{\#13}
\NormalTok{GaringerOzone.monthly.plot }\OtherTok{\textless{}{-}} \FunctionTok{ggplot}\NormalTok{(GaringerOzone.monthly, }\FunctionTok{aes}\NormalTok{(}\AttributeTok{y =}\NormalTok{ meanOzoneConcentration, }\AttributeTok{x =}\NormalTok{ firstmonth)) }\SpecialCharTok{+}
  \FunctionTok{geom\_point}\NormalTok{() }\SpecialCharTok{+}
  \FunctionTok{xlab}\NormalTok{(}\StringTok{"Date"}\NormalTok{) }\SpecialCharTok{+}
  \FunctionTok{ylab}\NormalTok{(}\StringTok{"Mean Monthly Ozone Concentration (ppm)"}\NormalTok{) }\SpecialCharTok{+}
  \FunctionTok{geom\_line}\NormalTok{()}
\FunctionTok{print}\NormalTok{(GaringerOzone.monthly.plot)}
\end{Highlighting}
\end{Shaded}

\includegraphics{EmmaChilds_A08_TimeSeries_files/figure-latex/unnamed-chunk-10-1.pdf}

\begin{enumerate}
\def\labelenumi{\arabic{enumi}.}
\setcounter{enumi}{13}
\tightlist
\item
  To accompany your graph, summarize your results in context of the
  research question. Include output from the statistical test in
  parentheses at the end of your sentence. Feel free to use multiple
  sentences in your interpretation.
\end{enumerate}

\begin{quote}
Answer:
\end{quote}

\begin{enumerate}
\def\labelenumi{\arabic{enumi}.}
\setcounter{enumi}{14}
\item
  Subtract the seasonal component from the
  \texttt{GaringerOzone.monthly.ts}. Hint: Look at how we extracted the
  series components for the EnoDischarge on the lesson Rmd file.
\item
  Run the Mann Kendall test on the non-seasonal Ozone monthly series.
  Compare the results with the ones obtained with the Seasonal Mann
  Kendall on the complete series.
\end{enumerate}

\begin{Shaded}
\begin{Highlighting}[]
\CommentTok{\#15}
\NormalTok{GaringerOzone.Components }\OtherTok{\textless{}{-}} \FunctionTok{as.data.frame}\NormalTok{(GOzone\_monthly\_decomposed}\SpecialCharTok{$}\NormalTok{time.series[,}\DecValTok{2}\SpecialCharTok{:}\DecValTok{3}\NormalTok{])}

\NormalTok{GaringerOzone.Components }\OtherTok{\textless{}{-}}\NormalTok{ GaringerOzone.Components }\SpecialCharTok{\%\textgreater{}\%}
  \FunctionTok{mutate}\NormalTok{(}\AttributeTok{data =}\NormalTok{ trend }\SpecialCharTok{+}\NormalTok{ remainder) }\SpecialCharTok{\%\textgreater{}\%}
  \FunctionTok{mutate}\NormalTok{(}\AttributeTok{date =}\NormalTok{ GaringerOzone.monthly}\SpecialCharTok{$}\NormalTok{firstmonth) }\SpecialCharTok{\%\textgreater{}\%}
  \FunctionTok{select}\NormalTok{(data, date)}
        
\NormalTok{GaringerOzone.Components.ts }\OtherTok{\textless{}{-}} \FunctionTok{ts}\NormalTok{(GaringerOzone.Components}\SpecialCharTok{$}\NormalTok{data, }\AttributeTok{start=}\FunctionTok{c}\NormalTok{(}\DecValTok{2010}\NormalTok{,}\DecValTok{1}\NormalTok{),}\AttributeTok{frequency=}\DecValTok{12}\NormalTok{)}

\CommentTok{\#16}
\NormalTok{GaringerOzone.MK }\OtherTok{\textless{}{-}} \FunctionTok{MannKendall}\NormalTok{(GaringerOzone.Components.ts)}
\FunctionTok{summary}\NormalTok{(GaringerOzone.MK)}
\end{Highlighting}
\end{Shaded}

\begin{verbatim}
## Score =  -1179 , Var(Score) = 194365.7
## denominator =  7139.5
## tau = -0.165, 2-sided pvalue =0.0075402
\end{verbatim}

\begin{quote}
Answer:
\end{quote}

\end{document}
